\documentclass[]{beamer}
\usepackage{tikz}
\usepackage{listings}
\usepackage{lmodern}

\usetikzlibrary{shapes.geometric, arrows}
\lstset{
  language={R},
  basicstyle=\ttfamily\small
}


%% The style for the flow chart
\tikzstyle{process} = [rectangle, text width=3cm, minimum height=1cm, text centered, draw=black, fill=orange!30]
\tikzstyle{arrow} = [thick,->,>=stealth]


\begin{document}

\begin{frame}
  \frametitle{Why programming in a data mining class?}
 
  \begin{block}{No two data challanges are alike}
    \begin{itemize}[<+->]
    \item Different sources (web, images, documents)
    \item Custom implementations (of the shelf solutions are not
      enough)
    \item Presentation is never the same (document, support system,
      images)
    \item Repeatable
    \end{itemize}  
  \end{block}
\end{frame}


\begin{frame}
  \frametitle{What programming language to choose?}
  
  \begin{columns}
    \begin{column}{0.5\textwidth}
      \begin{block}{Generic languages}
        \begin{itemize}
        \item Java 
        \item C/C++
        \item .NET 
        \item Fortran
        \item Python 
        \item Lua
        \end{itemize}
      \end{block}
    \end{column}
    \begin{column}{0.5\textwidth}
      \begin{block}{Specialized langauges}
        \begin{itemize}
        \item SPSS
        \item Matlab
        \item Octave
        \item GNU/S or R
        \end{itemize}
      \end{block}
    \end{column}
  \end{columns}  
\end{frame}


\begin{frame}
  \frametitle{Why use R?}

  R is Free (as in freedom and gratis) and widely accepted

  \begin{itemize}
  \item (pros) Public domain implementation
  \item (pros) Simple to use
  \item (pros) R has the best plots.
  \item (pros) Great environment (save/load)
  \item (pros) Mixed object oriented and functional
  \item (cons) R is slow
  \item (cons) R is weird
  \item (cons) R is slow
  \end{itemize}
\end{frame}

\begin{frame}
  \frametitle{Why use R?}
  
  \begin{itemize}
  \item R is polymorphic, which means that a function (say
    \texttt{plot}) can be applied to many objects (say a
    \texttt{matrix}, \texttt{list} or \texttt{song}) and produce a
    particular output depending on the input.

  \item R is object oriented, meaning that multiple properties of an
    object can be grouped, e.g., a function for producing a linear
    model can output a single linear regression object, containing the
    parameters and e.g., residuals and standard error.
    
  \item R is functional, with many functions applicable to sets of
    items, or single elements (e.g., \texttt{lapply(list(1:10, 2:20), median)})
  \end{itemize}
\end{frame}


\begin{frame}
  \frametitle{How to run R?}
  Download and install in a way that is suitable for your platform.

  \begin{itemize}
  \item \textbf{For GNU/Linux}, use your package manager
    (\texttt{apt}, \texttt{yum}, etc.) and install R (and possibly
    R-studio).

  \item \textbf{For Windows}, download a suitable R-installer from
    \url{http://r-project.org} and R-studio from
    \url{http://rstudio.com}.

  \item \textbf{For Mac}, download and install R using homebrew, and/or
    R-studio from \url{http://rstudio.com}.

    
  \item (For the tech-savvy, install Emacs Speak Statistics for a
    great environment.)

  \end{itemize}
\end{frame}

%   \begin{frame}{Statistical analysis value chain}
%     \begin{center}
%       \begin{figure}
%         \centering
%         \begin{tikzpicture}[scale=0.7,  transform shape, node distance=2cm]
%           \node (raw)  [process]                {Raw data}; %
%           \node (tech) [process, below of=raw]  {Technically Correct Data}; %
%           \node (cons) [process, below of=tech] {Consistent Data};  %
%           \node (stat) [process, below of=cons] {Statistical Results}; %
%           \node (form) [process, below of=stat] {Formatted  Output}; %
          
%           %% Connections
%           \draw [arrow] (raw)  -- node[anchor=west] {Type checking, normalizing, data types}   (tech); %
%           \draw [arrow] (tech) -- node[anchor=west] {Fix types, impute, remove, compute}       (cons); %
%           \draw [arrow] (cons) -- node[anchor=west] {Estimate, derive, analyze}                (stat); %
%           \draw [arrow] (stat) -- node[anchor=west] {Plot, tabulate, present}                  (form); %
%         \end{tikzpicture}
%         \caption{Statistical analysis value chain (adapted from \cite{})}
%       \end{figure}
%     \end{center}
%   \end{frame}


%   \begin{frame}
%     \frametitle{Raw data}
    
%     Raw data can be any source of information that can be processed in
%     a computing device represented in any format. Examples of raw data include:
    
%     \begin{itemize}
%     \item Text documents
%     \item Chemical compounds (perhaps represented as graphs)
%     \item Usage data (e.g., log files)
%     \item Images
%     \item Speech
%     \end{itemize}
%   \end{frame}

%   \begin{frame}{From Raw data to Technically Correct Data}

%     \begin{block}{A dataset}
%       In this tutorial we will assume that data is given in a
%       ``rectangular format'' where each \textbf{row} depicts a
%       \emph{real world object} of interest (a document, a chemical
%       compound or an image) and the \textbf{columns} denote a \emph{particular
%         variable}.      
%     \end{block}


%     \begin{block}{Technically correct data}
%       Data is technically correct if each value
%       \begin{enumerate}
%       \item can be directly identified as belonging to a particular
%         category of values (e.g., number, date)
%       \item each value is stored in a data type that represents the
%         domain of a real-world variable
%       \end{enumerate}

%       Texts are stored as texts, numbers as numbers, dates as dates,
%       time as time etc.
%     \end{block}
%   \end{frame}

%   \begin{frame}<handout:0>
%     \frametitle{A dataset in R}
%     \begin{center}
%       Example 1
%     \end{center}
%   \end{frame}

%   \begin{frame}<beamer:0>[containsverbatim]
%     \frametitle{A dataset in R (Example 1)}
% \begin{lstlisting}[language=R]
% > dataset <- data.frame("x" = c(10, 20, 30), 
%                         "y" = c("a", "b", "c"))
%    x y
% 1 10 a
% 2 20 b
% 3 30 c
% \end{lstlisting}
%   \end{frame}

\end{document}
%%% Local Variables:
%%% mode: latex
%%% TeX-master: t
%%% End:
