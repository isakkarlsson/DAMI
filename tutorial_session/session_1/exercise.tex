\documentclass{article}[11pt]
\usepackage{parskip}
\begin{document}


Lets start R!

\begin{verbatim}
$ R
R version 3.2.3 (2015-12-10) -- "Wooden Christmas-Tree"
Copyright (C) 2015 The R Foundation for Statistical Computing
Platform: x86_64-apple-darwin15.2.0 (64-bit)
....
Type 'q()' to quit R.
> 
\end{verbatim}

This will start the R interpreter (the machinery that interprets R code).

Let's now make a simple one dimensional data set - in R a
\emph{vector} - consisting of a few numbers.

\begin{verbatim}
> x <- c(10, 20, 30, 40, 50)
> x -> y
> y = x
\end{verbatim}

The standard \emph{assignment operator} in R is \texttt{<-}. You can
also use the right-pointing version or the equal sign. For most
practical purposes, prefer the left-pointing assignment operator.

\texttt{c} is a standard R-function for \texttt{c}oncatenating vectors. In our case, we concatenate 5 numbers. Or, really, 5 one-element vectors.

\begin{verbatim}
> c(x, x, 3, 2, 1, 1:3)
 [1] 10 20 30 40 50 10 20 30 40 50  3  2  1  1  2  3
\end{verbatim}

We have now learned one more thing. How to print stuff to the
console. Writing a variable name, or more specifically - an expression
- in the interactive console and pressing enter will output the value
of the expression or variable.

\begin{verbatim}
> x
 [1] 10 20 30 40 50
\end{verbatim}

One can thus use, the R-interpreter as a calculator

\begin{verbatim}
> 1 + 1
 [1] 2
> sqrt(10)
 [1] 3.162278
> sqrt(c(1,2,-3))
 [1] 1.000000 1.414214      NaN
\end{verbatim}


\texttt{NaN} is used to represent the result of undefined numbers,
such as the square-root of a negative number. There are also
\texttt{Inf} and \texttt{-Inf}, which represents positive and negative infinity.

Try!


\begin{verbatim}
> -exp(1)**10000
 [1] -Inf
> 1/0
 [1] Inf
> 1 < -Inf
 [1] FALSE
> 1 < Inf
 [1] TRUE
> 1 < NaN
 [1] NA
\end{verbatim}

A new constant appeared. \texttt{NA} is used to represent the absence
of value. Similar to \texttt{null} in Java or \texttt{None} in python.

Note that many of the built in functions in R are designed to cope
with \texttt{NA}-values. For example, \texttt{NA+1} results in
\texttt{NA}.









\end{document}



%%% Local Variables:
%%% mode: latex
%%% TeX-master: t
%%% End:
