\documentclass{article}[11pt]
\usepackage{parskip}
\usepackage[utf8]{inputenc}
\begin{document}


Lets start R!

\begin{verbatim}
$ R
R version 3.2.3 (2015-12-10) -- "Wooden Christmas-Tree"
Copyright (C) 2015 The R Foundation for Statistical Computing
Platform: x86_64-apple-darwin15.2.0 (64-bit)
....
Type 'q()' to quit R.
> 
\end{verbatim}

This will start the R interpreter (the machinery that interprets R code).

Let's now make a simple one dimensional data set - in R a
\emph{vector} - consisting of a few numbers.

\begin{verbatim}
> x <- c(10, 20, 30, 40, 50
> x -> y
> y = x
\end{verbatim}

The standard \emph{assignment operator} in R is \texttt{<-}. You can
also use the right-pointing version or the equal sign. For most
practical purposes, prefer the left-pointing assignment operator.

\texttt{c} is a standard R-function for \texttt{c}oncatenating vectors. In our case, we concatenate 5 numbers. Or, really, 5 one-element vectors.

\begin{verbatim}
> c(x, x, 3, 2, 1, 1:3)
 [1] 10 20 30 40 50 10 20 30 40 50  3  2  1  1  2  3
\end{verbatim}

We have now learned one more thing. How to print stuff to the
console. Writing a variable name, or more specifically - an expression
- in the interactive console and pressing enter will output the value
of the expression or variable. Note that the \texttt{[1]} here denotes
the first element index of the line (and yes - in R indexing starts
with 1 and not 0).

\begin{verbatim}
> x
 [1] 10 20 30 40 50
\end{verbatim}

One can thus use, the R-interpreter as a calculator

\begin{verbatim}
> 1 + 1
 [1] 2
> sqrt(10)
 [1] 3.162278
> sqrt(c(1,2,-3))
 [1] 1.000000 1.414214      NaN
\end{verbatim}


\texttt{NaN} is used to represent the result of undefined numbers,
such as the square-root of a negative number. There are also
\texttt{Inf} and \texttt{-Inf}, which represents positive and negative infinity.

Try!

\begin{verbatim}
> -exp(1)**10000
 [1] -Inf
> 1/0
 [1] Inf
> 1 < -Inf
 [1] FALSE
> 1 < Inf
 [1] TRUE
> 1 < NaN
 [1] NA
\end{verbatim}

A new constant appeared. \texttt{NA} is used to represent the absence
of value. Similar to \texttt{null} in Java or \texttt{None} in python.

Note that many of the built in functions in R are designed to cope
with \texttt{NA}-values. For example, \texttt{NA+1} results in
\texttt{NA}.


Going back to our old vector \texttt{x}, we can get the individual
items using the \texttt{[]} operator.

\begin{verbatim}
> x[3] # square brackets for indexing
 [1] 30

> x <- runif(5,min = 0, max = 20)
[1]  9.783508 19.945672 12.984764  0.880098 10.158592

> x < 0
> x[x < 0]
> x[x < 0] <- NA
> x[is.na(x)]
> x[1:10]
> x[x < 0]
\end{verbatim}

We can find the \texttt{mean} and \texttt{median} of the vector

\begin{verbatim}
> mean(x)
 [1] 30
> median(x)
 [1] 30
> x.mean <- mean(x) # save the mean
> cumsum(x)
> summary(x)
\end{verbatim}

Here, note that we are using \texttt{\#} to denote comments.

Remember, the main advantage of R is for data processing. For
learning, R provides a list of educational datasets. We can list them
with the \texttt{data()} function.

\begin{verbatim}
> data()
....
\end{verbatim}

Here, you will enter a \texttt{less}-like environment, where you can
scroll up with your arrow keys. You quit by entering \texttt{q}.

Lets try one of them.

\begin{verbatim}
> Nile
Time Series:
Start = 1871 
End = 1970 
Frequency = 1 
  [1] 1120 1160  963 1210 1160 1160  813 1230 1370 1140  995  935 1110  994 1020
 [16]  960 1180  799  958 1140 1100 1210 1150 1250 1260 1220 1030 1100  774  840
 [31]  874  694  940  833  701  916  692 1020 1050  969  831  726  456  824  702
 [46] 1120 1100  832  764  821  768  845  864  862  698  845  744  796 1040  759
 [61]  781  865  845  944  984  897  822 1010  771  676  649  846  812  742  801
 [76] 1040  860  874  848  890  744  749  838 1050  918  986  797  923  975  815
 [91] 1020  906  901 1170  912  746  919  718  714  740
\end{verbatim}

As we can see, we are dealing with a time series. Lets figure out the
mean and the standard deviation.

\begin{verbatim}
> mean(Nile)
 [1] 919.35
> sd(Nile)
 [1] 169.2275
\end{verbatim}

We can also plot a histogram.

\begin{verbatim}
> hist(Nile)
\end{verbatim}

You can now quit R.

\begin{verbatim}
> q()
Save workspace image? [y/n/c]: 
\end{verbatim}

It gives you the option to store an image of your current R-session,
with variables and such stored for later which can be useful when
working on something for extended periods of time. However, be careful
and try to store your scripts in files.

In many cases the interactive mode is fine. However, if larger chunks
of code needs to be written or if it must be shared among several
coders, defining code files and functions is necessary. Lets define a
function for counting positive integers in a vector.


\begin{verbatim}
count.even <- function(x) {
  c <- 0
  for (e in x) {
    if(e %% 2 == 0) {
       c <- c + 1
    }
  }

  return(c)
}

count.even.vec <- function(x) {
  sum(x %% 2 == 0)
}
# we can investigate the speed difference using the microbenchmark
# package. 
install.packages(microbenchmark)
library(microbenchmark)
library(ggplot2)
autoplot(microbenchmark(count.even(...), count.even.vec(...)))
\end{verbatim}

First we assign a function to the variable name
\texttt{count.even}. We start by assigning the initial value to
\texttt{c}, and for each \texttt{e}lement in the vector \texttt{x}
check if it is even. If so we increment our counter. Finally we return
the result.

In the second example, we define the function in a vectorized manner,
which simplifies and speeds it up.

\subsection*{Data Structures in R}

\begin{description}
\item[Vector] We have seen that. A series of elements of the same
  type. In R, these are \texttt{logical}, \texttt{numeric},
  \texttt{character}. We can convert between them.

\begin{verbatim}
> x <- c(1,2,3,4); class(x)
> as.numeric(c(TRUE, FALSE, NA))
[1]  1  0 NA
\end{verbatim}
\item[List] A list is more akin to an object in Java or Python.

\begin{verbatim}
> list(a=10, b=20)
$a
[1] 10

$b
[1] 20
\end{verbatim}. Elements can be accessed by using \texttt{\$}, e.g., \texttt{x\$a}.
\item[Matrix] A matrix is two-dimensional data structure with rows and
  columns. Mostly used for linear algebra routines. \texttt{rbind}
  stands for row-bind and concatenates vectors as rows.

\begin{verbatim}
> m <- rbind(c(1,2,3), c(3,1,2))
> m
     [,1] [,2] [,3]
[1,]    1    2    3
[2,]    3    1    2
> m %*% c(1,2,3)
     [,1]
[1,]   14
[2,]   14
> m * 2
     [,1] [,2] [,3]
[1,]    2    4    6
[2,]    6    2    6
\end{verbatim}. \texttt{\%*\%} can be used to perform matrix-matrix multiplication and \texttt{*} for elementwise multiplication.
\item[Data Frame] A data frame is list of vectors as columns. In contrast to a matrix, a data frame can have different types for each column.

\begin{verbatim}
> ChickWeight
    weight Time Chick Diet
1       42    0     1    1
2       51    2     1    1
3       59    4     1    1
4       64    6     1    1
5       76    8     1    1
> ChickWeight$Time
> ChickWeight$Time
  [1]  0  2  4  6  8
\end{verbatim}
\end{description}

\subsection*{Larger example}

In this example we will learn how to load a csv-file and mess around
with the resulting data frame.

\begin{verbatim}
> x <- read.table("smoking_data.csv")
# ... realize error ...
> x <- read.table("smoking_data.csv", sep=",")
# ... realize error ...
> x <- read.table("smoking_data.csv", 
                  sep=",", 
                  skip=1, 
                  col.names=c("age", "smoker", "cancer"))
> x
   age smoker cancer
1   55      0      0
2   28      0      0
3   65      1      0
4   46      0      1
5   86      1      1
6   56      1      1
7   85      0      0
8   33      0      0
9   21      1      0
10  42      1      1

> x <- read.csv("smoking_data.csv") # simpler
> class(x) # confirm class
[1] "data.frame"
> head(x)
  Age Smoker Cancer
1  55      0      0
2  28     NA      0
3  65      1      0
4  46      0      1
5  86      1      1
6  56      1      1

# .... some cleaning ....
> x <- na.omit(x) # remove rows with NA-values
> x$Smoker <- as.factor(x$Smoker)
> x$Cancer <- as.factor(x$Cancer)
> x$Age <- as.numeric(x$Age)
\end{verbatim}

Let us make some plots. For this we need to install a package. This is
done with the \texttt{install.packages} function. So, lets
\texttt{install.packages(``ggplot2'')}.


\begin{verbatim}
# one variable discrete
> 
> ggplot(x, aes(Smoker)) + geom_bar()

# one variable continuous
> ggplot(x, aes(Age)) + geom_dotplot()
> ggplot(x, aes(Age)) + geom_density()

# two variables, factor and continuous
> ggplot(x, aes(x=smoker, y=age)) + geom_boxplot()
\end{verbatim}

In this example, \texttt{aes} stands for aesthetics. ggplot is the
grammar of graphics. It builds a plot from a few components,
\textbf{data}, \textbf{geoms} and a \textbf{coordinate system}. In
this example, data is our smoker dataset, and maps x to the
smoker column. It then adds the visual marks as a bar-plot.

Anyway. Would it not be interesting to see how the age and smoker
factor affect the probability of getting cancer? This is where the
logistic regression model enters the stage. It uses some data to
predict the likelihood of an event.

To do this, we will use the \texttt{glm} function. It takes a formula
(read more \texttt{?~}) on the form \texttt{response ~ predictor
  ...}. In our case, we would like to predict \texttt{Cancer} based on
\texttt{Age} and \texttt{Smoker}.


\begin{verbatim}
> m <- glm(Cancer ~ Age + Smoker, family=binomial, data=x)
> coefficients(m)
(Intercept)         age     smoker1 
-1.97939736  0.01568492  1.54928419 

# raising e to the power of the coefficients give us the odds ratio
> exp(coefficients(m))
(Intercept)         age     smoker1 
  0.1381525   1.0158086   4.7080989 
\end{verbatim}

Thus, in this example we can conclude that higher age gives a $1.015$
greater odds of having cancer; and smoking $4.7$ times greater odds.

We can also plot various statistics regarding the model using
\texttt{plot(m)}.

\subsection*{R statements}

\begin{verbatim}
# for loop
for(x in c(1,2,3)){
  print(x)
}

# while loop
i <- 0
while(i < 10) {
  print(i)
  i <- i+2
}

# if/else
a <- 10
if(a < 10) {
  print("<")
} else if(a == 10) {
  print("=")
} else {
  print(">")
}

# conditions
1 > 2
2 < 1
2 >= 2
3 <= 10
TRUE && TRUE
FALSE || TRUE
c(TRUE, TRUE) & c(TRUE, FALSE)
c(TRUE, TRUE) | c(TRUE, FALSE)

# type conversions
str(c(1,2,3))
as.logical(c(1,0,10))
\end{verbatim}

Functions are objects.

\begin{verbatim}
> f <- function(a) return(a*2)

# you don't have to specify return. Last executed statement is returned
> multiplier <- function(a) function(x) x*a
> multiplier(10)(2)
[1] 20

# values are always supplied as copies
> f <- function(x) { x[1] <- 10 }
> x <- c(1,2,3)
> f(x)
> x
[1] 1 2 3

# we solve this by reassignment
> f <- function(x) { x[1] <- 10; return(x) }
> x <- f(x)
> x
[1] 10  2  3

# if we have multiple parameters, return a list
> f <- function(a, b) { a[1] <- 10; b[1] <- 10; return(list(a=a,b=b))}
> f(x,x)
$a
[1] 10  2  3

$b
[1] 10  2  3

# default values
> multiplier <- function(a=2) function(x) x*a
> multiplier()(10)
[1] 20

# you can create new operators
> "%concat%" <- function(a,b, sep=" ") paste(a,b,sep=" ")
> "hello" %concat% "world"
[1] "hello world"

> "%conj%" <- function(a,b) c(a,b)
> 1 %conj% 2 %conj% 3
[1] 1 2 3
\end{verbatim}

\subsection*{Input/output}

\begin{verbatim}
# read from cmd
> s <- scan()
1: 10
2: 
Read 1 item
> s
[1] 10

> s <- readline()
hello world

# printing

>  print("hello world")
[1] "hello world"

> cat(1,2,3,"hello", "world")
1 2 3 hello world

> read.table("smoking_data.csv", sep=",", header=TRUE)
> read.csv("smoking_data.csv")
> M <- as.matrix(read.table("smoking_data.csv", sep=",", header=TRUE))
> t(M) %*% M
         Age Smoker Cancer
Age    31201     NA    230
Smoker    NA     NA     NA
Cancer   230     NA      4
\end{verbatim}

\subsection*{Object orient programming}


\begin{verbatim}
# create a class
> setClass("student", representation(
+                         name="character", 
+                         score="numeric"))

> joe <- new("student", name="Joe", score=1.0)
> joe
An object of class "student"
Slot "name":
[1] "Joe"

Slot "score":
[1] 1
> joe$score 
Error
> joe@score

# we can define a new show method
> setMethod("show", "student", function(self) {
+   cat(self@name, "has score", self@score, "\n")
+ })
> joe
Joe has score 1
\end{verbatim}


\subsection*{More examples}

\begin{verbatim}
> ggplot(diamonds, aes(cut, price)) + geom_boxplot()
> ggplot(diamonds, aes(color, price)) + geom_boxplot()
> ggplot(diamonds, aes(clarity, price)) + geom_boxplot()

> ggplot(diamonds, aes(carat, price)) + geom_point()
> ggplot(diamonds, aes(carat, price)) + geom_point(aes(color=cut))

# The log-transformation is particularly useful here because it makes
# the pattern linear, and linear patterns are the easiest to work
# with. Let’s take the next step and remove that strong linear
# pattern. We first make the pattern explicit by fitting a model:

> diamonds.carat.less <- diamonds[diamonds$carat < 2.5,] # select lower carat
> diamonds.carat.less$logPrice <- log2(diamonds.carat.less$price)
> diamonds.carat.less$logCarat <- log2(diamonds.carat.less$carat)
> ggplot(diamonds, aes(logCarat, logPrice)) + geom_point()

> diamonds.model <- lm(logPrice ~ logCarat, diamonds.carat.less)

> extra.carat <- log2(seq(from=0, to=3, length=10000))
> predictions = data.frame(
   price = 2**predict(diamonds.model, data.frame(logCarat=extra.carat)),
   carat=2**carat
)

> ggplot(diamonds.carat.less, aes(carat, price))
   + geom_point() 
   + geom_line(data=predictions)
\end{verbatim}

\end{document}



%%% Local Variables:
%%% mode: latex
%%% TeX-master: t
%%% End:
