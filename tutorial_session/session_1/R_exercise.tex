\documentclass[11pt]{article}

\title{DAMI, R exercises}

\begin{document}

\maketitle


\subsection*{Load data}
You have been given a file \texttt{students.csv}. Load this file using
a suitable R-function.

\noindent \textbf{NOTE:} Before loading the file, make sure you
understand your working directory. You can investigate your current
working directory with \texttt{getwd()} and change it with
\texttt{setwd()}. You can list the files in your working directory
with \texttt{dir()}.

\subsection*{Explore the data}
\begin{enumerate}
\item Use the \texttt{head()} function to view the first rows of your
  data frame. Play with the \texttt{n}-argument. You can also use the
  \texttt{tail()} function to view the last rows. Use \texttt{?tail}
  find the argument used to specify the number of rows shown to the
  screen. Try the \texttt{names} function. What does it do?

\item You should also summarize each column of the data frame
  (remember \texttt{\$} is used to select columns). Is there a simpler
  way?

\item For each column, figure out the \texttt{mean}, \texttt{median},
  \texttt{quantile} and \texttt{sd} for each column.

\item Use the \texttt{table} function to count the female and male
  students, e.g., \texttt{table(students\$gender)}. Now using,
  \texttt{?table}, figure out how to count the number of male and
  female students in \texttt{stockholm} and \texttt{gothenburg}
  respectively.
\end{enumerate}
\subsection*{Plotting}
\begin{enumerate}
\item Plot a histogram of the \texttt{age}-variable (\texttt{hist}),
  now try and reproduce this plot using \texttt{ggplot2}.


 
\item Try reproducing the plot obtained by running the following
  command:

  \noindent\texttt{boxplot(students\$height \textasciitilde
    students\$gender)}
  
  using \texttt{ggplot2} (hint, use \texttt{geom\_boxplot()}). Does
  the same trend (there is a large difference in height between
  genders) exist for different campuses?

\item Make a scatter plot of the \texttt{shoesize} vs the
  \texttt{height}. For example:

  \noindent\texttt{plot(students\$shoesize, students\$height)}. 

  Try reproducing this plot using \texttt{ggplot2} (hint, use
  \texttt{geom\_point()}). 

  What if we want to differentiate between male and female students in
  the plot? Use \texttt{plot()} and try to colorize each point
  according to gender. Try to do the same using \texttt{ggplot}.
\end{enumerate}

\subsection*{Extract subsets of the data}

Remember the indexing operator \texttt{[]}. For data frames it works as follows:

\begin{verbatim}
> students[c(1,2)]   # selects the first and second column
> students[c(1,2), ] # selects the first and second row 
                     # (note the trailing comma)
> students[c("height", "gender")] # selects the specified columns
\end{verbatim}



\begin{enumerate}
\item Running \texttt{students[students\$gender == "male"]} will
  fail. Why is that? Fix the error (hint, we are selecting columns).
\item Select only male students and store them in a variable called \texttt{students.male}.
\item Select only female students and store them in a variable called
  \texttt{students.female}.
\item Use the manual to understand how the \texttt{which()} function
  works. Use this function to select male and female students again.
\item Partition the data frame into two groups, male students whose
  height is above the median (for male students). And female students
  who are above the median height (for male students).
\end{enumerate}

\end{document}

%%% Local Variables:
%%% mode: latex
%%% TeX-master: t
%%% End:
